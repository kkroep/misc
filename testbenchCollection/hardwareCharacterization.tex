\section*{Hardware Characterization}

\subsection*{Required measurements}
\begin{lstlisting}
http://ra.ziti.uni-heidelberg.de/pages/student_work/
seminar/ws0304/richard_sohnius/praesentation.pdf
\end{lstlisting}


Transition times. Characterized and measured between $30\,\%$ and $70\,\%$. Standardized reporting between $10\,\%$ and $90\,\%$.


\begin{s_itemize} 
\item area
\item power
\item timing contstraints (setup/Hold time, recovery/removal time)
\item propagation delay time
\item input capacitance
\end{s_itemize} 
%
%
\textbf{Power}:
\begin{s_itemize}
	\item Dynamic power (switching power)
	\item Static power (leakage power)
	\item Passive power (internal power. power used by sequential cells when inputs change without output change)
\end{s_itemize}
%
%
\textbf{Delay}:
\begin{lstlisting}
https://www.csee.umbc.edu/~cpatel2/links/641/slides/lect05_LIB.pdf
\end{lstlisting}

\begin{s_itemize}
\item sum of intrinsic delay, slope delay, transition delay and connect delay
\item intrinsic delay is fixed value independent of surroundings
\item slope delay is produced by the slew of the input signal
\item transition delay is the time required to change the capacitance of the next stage input pins
\item connect delay is delay caused by the wire between output and next input pins
\end{s_itemize}

Delay model of Liberate (.lib) is the CMOS Non-Linear Delay Model.
\begin{s_itemize}
\item Delay and transition time are modeled as function of Input slew and Output load
\item Data is stored in a 2D LUT (look-up table)
\item Intermediate values are interpolated
\item Data points are usually not equidistant
\end{s_itemize}





\subsection*{solutions}
feedback loop with a counter?